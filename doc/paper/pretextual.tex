% Os elementos a seguir aparecerão antes do início do artigo em si.

% Espaçamento de 1,5cm por linha
\OnehalfSpacing

% Capa
\imprimircapa

% Folha de rosto
\imprimirfolhaderosto

% Folha de aprovação
% COMANDO EXCLUSIVO DA ABNTEX2UNIFEI.
\imprimirfolhadeaprovacao{Este Trabalho de Pesquisa foi julgado, como requisito parcial, para aprovação na disciplina Trabalho Final de Graduação da Engenharia da Computação da Universidade Federal de Itajubá – \textit{campus} Itabira.}


% Espaço reservado a dedicatórias
%\begin{dedicatoria}
%\null
%\vfill
%Espaço reservado à dedicatória, elemento opcional destinado a homenagear %pessoas importantes na vida do autor do trabalho.
%\end{dedicatoria}


% Agradecimentos
\begin{agradecimentos}[Agradecimentos]
Aos professores e demais do corpo discente da faculdade pelo conhecimento proporcionado ao longo de vários anos. Conhecimento técnico, profissional e também pessoal, que auxiliou na forma como entendemos e enxergamos o mundo -- uma via de mão dupla. Também ao nosso professor e orientador Paulo, pela paciência e orientação a nós e ao projeto, inclusive pelos conteúdos lecionados que se relacionam com o tema deste trabalho.
\end{agradecimentos}

% Epígrafe
%\begin{epigrafe}
%\null
%\vfill
%Espaço reservado à epígrafe, elemento opcional, elaborado conforme a ABNT NBR 10520, em que se transcreve uma citação literal, com autoria, referente ao assunto abordado no trabalho.
%\end{epigrafe}


% Resumo
\begin{resumo}
O \textit{Bitcoin Core}, nodo mais popular de \textit{Bitcoin}, está sendo modificado para que possua uma menor quantidade de bloqueios de execução concorrente. Decorrente deste tipo de modificação ser delicada e da necessidade da execução paralela, que permitiria maior uso da capacidade total dos computadores de múltiplos núcleos, é desejável que exista uma implementação alternativa de nodo que tenha garantias de segurança de memória, uma vez que a detecção e a correção posterior dos erros de execução paralela seriam custosas. Tais garantias são inerentes à linguagem \textit{Rust}, que com conceitos incomuns, oferece ferramentas que, em tempo de compilação, previnem algumas classes de erros de memória, como \textit{data-races} e \textit{use-after-free}.

Foram compreendidos os conceitos gerais em torno do ecossistema do \textit{Bitcoin}, da sua \textit{Blockchain} e em torno do funcionamento de um nodo de uma rede \textit{P2P}, para que componentes de um nodo pudessem ser planejados, implementados e utilizados de tal forma que consigam realizar conexões com outros nodos da rede e oferecer uma estrutura inicial de um nodo gerenciável, de modo a facilitar o desenvolvimento de componentes posteriores, que poderão possuir as mesmas características desejáveis de assincronia e segurança.

Para facilitar a execução assíncrona dos componentes, foram implementados \textit{actors} do sistema \textit{Actor Model}, alguns dos quais possuem um comportamento definido por máquinas de estados, em uma topologia de canais de comunicação entre os \textit{actors}.

O programa foi compilado com sucesso e tal resultado indica fortes garantias de segurança, sendo isto devido, principalmente, às garantias inerentes ao compilador da linguagem de programação \textit{Rust}. Também foram feitos testes que corroboram com a assincronicidade dos componentes internos e do comportamento do nodo em relação ao processamento de informação e sua comunicação com a rede.

Palavras-chave: \textit{Actor Model}. \textit{Bitcoin}. Nodo. \textit{P2P}. \textit{Rust}.
\end{resumo}

\begin{resumo}[Abstract]
The Bitcoin Core, the most popular Bitcoin node, is being modified in order to have fewer thread blocks on concurrent excecution. Due to the delicated nature of this kind of modification, and also to the parallel execution requierement, which results in a higher usage of a multicore computer's total capacity, it is desired that there is an alternative node implementation that offers memory-safety guarantees when concurrent access are being made.

It was intended to understand the general concepts around the Bitcoin ecosystem, it's Blockchain and also around the operation of a P2P Node Network, so that components of a node could be planned, implemented and used in such that they would connect with other nodes from the network and also offer an initial structure of a manageable node, facilitating the development of later components that may have the same asynchrony and security requirements.

To facilitate the asynchronous execution of the components, actors from the Actor Model system were implemented, some of which had a behavior defined by state machines, also in a communication channel topology between themselves.

The software has been successfully compiled and therefore implies a strong security guarantee, mainly due to the inertors of the compiler of the Rust programming language. Tests were also carried out to corroborate to the internal components' asynchronicity and the node's behavior in relation to information processing and it's network communication.

Keywords: Actor Model. Bitcoin. Node. P2P. Rust.

\end{resumo}


% Lista de Figuras
\pdfbookmark[0]{\listfigurename}{lof}
\listoffigures*
\cleardoublepage

% Lista de Tabelas
%\pdfbookmark[0]{\listtablename}{lot}
%\listoftables*
%\cleardoublepage

%Lista de Abreviaturas e Siglas
\begin{siglas}
    \item[Arc] \textit{Atomic Reference Counter}
    \item[BTC] \textit{Bitcoin}
    \item[CLI] \textit{Command-Line Interface}
    \item[CPU] \textit{Central Processing Unit}
    \item[IPC] \textit{Inter Process Communication}
    \item[MPSC] \textit{Multiple Producer Single Consumer}
    \item[Mutex] \textit{Mutual Exclusion}
    \item[P2P] \textit{Peer-to-Peer}
    \item[QRCode] \textit{Quick Response Code}
    \item[Rc] \textit{Reference Counter}
    \item[RIPEMD] \textit{RACE Integrity Primitives Evaluation Message Digest}
    \item[RwLock] \textit{Reader Writer Lock}
    \item[SHA] \textit{Secure Hash Algorithm}
    \item[SPV] \textit{Simple Payment Verification}
    \item[TCP] \textit{Transmission Control Protocol}
    \item[UTXO] \textit{Unspent Transaction Output}
\end{siglas}

% Lista de Símbolos
%\begin{simbolos}
%  \item[$ \Gamma $] Letra grega Gama
%  \item[$ \Lambda $] Lambda
%  \item[$ \zeta $] Letra grega minúscula zeta
%  \item[$ \in $] Pertence
%  \item[O(n)] Ordem de um algoritmo
%  \item[©] Copyright
%\end{simbolos}

% Sumário
\pdfbookmark[0]{\contentsname}{toc}
\tableofcontents*
\cleardoublepage